\documentclass{article}

\usepackage[myheadings]{fullpage}
\usepackage[french]{babel}
\usepackage{fancyhdr}
\usepackage[utf8]{inputenc}
\usepackage{indentfirst}
\usepackage{enumerate}
\usepackage{amsmath}
\usepackage{listings}
\usepackage{graphicx}
\usepackage{float}

\newcommand{\titleinfo}{[INFO0054] Programmation Fonctionnelle}
\title{Programmation Fonctionnelle - Loi de Zipf}
\author{Benoît Dardenne}
\date{Avril 2013}

\pagestyle{fancy}

\lhead{Benoît Dardenne \\ 3\textsuperscript{ème} Bac. Sc. Info.}
\rhead{Avril 2013 \\ \titleinfo}

\begin{document}
\maketitle


\subsection*{Structures de données}

\paragraph{} Une structure de données assez adaptée au genre de manipulations demandées ici est le \emph{trie}.\\
Pour ce travail, j'ai choisi de représenter un trie comme une liste de nœuds, 
et un nœud comme une liste 
composée d'un caractère (appelé \emph{clef}), d'un entier (appelé \emph{valeur}) et d'un trie. 
Pour chaque nœud, l'entier associé correspond
au nombre de chaînes contenues dans le trie passant par ce nœud, et le trie associé est un trie 
contenant toutes les chaînes commençant par le catactère \emph{clef}. 
Ainsi, pour un chemin de nœuds dont les clefs sont \verb#s#, \verb#h# et \verb#e# 
la valeur associée à la clef \verb#e# donne le nombre de chaînes ayant pour préfixe \verb#she#
qui sont contenues dans le trie.
Pour dénoter un mot qui serait préfixe d'un autre (par exemple ``she'' et ``shells'') on utilise une
valeur-drapeau, ici \verb#eol#, indiquant une fin de mot.

\begin{figure}[H]
   \centering
   \includegraphics{stgraph.png}
   \label{fig:trie}
   \caption{Exemple de trie après insertions successives des mots : ``She sells sea shells by the sea shore"}
\end{figure}

\paragraph{} En insérant ainsi tous les mots d'un texte donné dans un trie, il suffit de parcourir ce trie et,
pour chaque nœud dont la clef est \verb#eol#, la valeur donne le nombre d'occurences d'un mot, déterminé 
par le chemin parcouru pour atteindre ce nœud.fdqf 
u
fdsf

\end{document}

