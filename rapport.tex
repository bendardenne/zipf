\documentclass{article}

\usepackage[myheadings]{fullpage}
\usepackage[french]{babel}
\usepackage{fancyhdr}
\usepackage[utf8]{inputenc}
\usepackage{indentfirst}
\usepackage{enumerate}
\usepackage{amsmath}
\usepackage{listings}
\usepackage{graphicx}
\usepackage{float}
\usepackage{tikz}

\newcommand{\titleinfo}{[INFO0054] Programmation Fonctionnelle}
\title{Programmation Fonctionnelle - Loi de Zipf}
\author{Benoît Dardenne}
\date{Avril 2013}

\pagestyle{fancy}

\lhead{Benoît Dardenne \\ 3\textsuperscript{ème} Bac. Sc. Info.}
\rhead{Avril 2013 \\ \titleinfo}

\begin{document}
\maketitle


\subsection*{Structures de données}
\paragraph{}
Une structure de données assez adaptée au genre de manipulations demandées ici est le \emph{trie}.\\
Pour ce travail, j'ai choisi de représenter un trie comme une liste de nœuds, 
et un nœud comme une liste 
composée d'un caractère (appelé \emph{clef}), d'un entier (appelé \emph{valeur}) et d'un trie.

\[
   trie := (node ~~ node ~~ node ~~ ...) 
\]
\[
   node := (clef ~~ valeur ~~ trie)
\]
 
Pour chaque trie, les clefs des nœuds sont uniques parmi l'ensemble des caractère alphabétiques 
minuscules. Un trie a donc un maximum de 26 nodes, si l'on ignore les caractères accentués et autres 
symboles ésotériques.

Insérer un mot dans un trie consiste à choisir le nœud dont la clef est correspond à la première lettre
de ce mot, incrémenter la valeur associée à cette clef et enfin insérer le mot privé de sa première lettre
dans le sous-trie du nœud. Pour chaque noeud $\alpha$ de clef $a$, l'entier associé correspond donc
au nombre de chaînes contenues dans le trie passant par ce nœud,
c'est-à-dire le nombre de mots qui ont pour préfixe la chaîne formée par $a$ et les clefs de 
tous les ancêtres de $\alpha$, 
et le sous-trie de $\alpha$ contient toutes les mots commençant par ce même préfixe. 

Pour dénoter un mot qui serait préfixe d'un autre (par exemple ``she'' et ``shells'') on utilise une
valeur-drapeau, ici \verb#eol#, indiquant une fin de mot.

En insérant ainsi tous les mots d'un texte donné dans un trie, il suffit de parcourir ce trie et,
pour chaque nœud dont la clef est \verb#eol#, la valeur donne le nombre d'occurrences d'un mot, déterminé 
par le chemin parcouru pour atteindre ce nœud.

\paragraph{}
Voici un petit exemple d'un trie dans lequel on a successivement inséré les mots de la phrase
``She sells sea shells by the sea shore'' :

\begin {figure}[H]
\begin{tikzpicture}[level/.style={sibling distance=60mm/#1}]
\node [circle,draw] {}
  child {node [circle,draw] {s,6}
    child {node [circle,draw] {h,3}
      child {node [circle,draw] {e,2}
         child{node [circle, draw] {l,1}
            child{node [circle, draw] {l,1}
               child{node [circle, draw] {s,1}
                  child{node {eol,1}}
                }
            }
         }
         child{node {eol,1}}
      } 
      child {node [circle,draw]{o,1}
         child {node [circle,draw]{r,1}
            child {node [circle,draw]{e,1}
               child{node {eol,1}}
            }
         }
      }
    }
    child {node [circle,draw] {e,3}
      child{node [circle, draw] {l,1}
         child{node [circle, draw] {l,1}
            child{node [circle, draw] {s,1}
               child{node {eol,1}}
            }
         }
      }
      child{node [circle, draw] {a,2}
         child{node {eol,2}}
      }
    }
  }
  child {node [circle,draw] {b,1}
    child {node [circle,draw] {y,1}
      child{node {eol,1}}
     }
  }
   child{ node [circle,draw] {t,1}
      child {node [circle, draw] {h,1}
         child {node [circle, draw] {e,1}
            child{node {eol,1}}
         }
      }
   }
;  

\end{tikzpicture}
   \label{fig:trie}
   \caption{Exemple de trie après insertions successives des mots : ``She sells sea shells by the sea shore"}
\end{figure}

\subsection*{Utilisation}
\paragraph{}
La fonction résolvant le problème demandé est la fonction \verb#top100#. Elle prend en paramètre 
unique le chemin du fichier à analyser et elle génère le graphique des fréquences avant de retourner
une liste de 100 paires pointées de la forme $(mot ~ . ~ frequence)$. 

\subsection*{Choix d'implémentation}
\begin{description}

\item [Définition d'un mot] J'ai choisi de considérer qu'un mot était toute suite de caractères
alphabétiques. Ceci exclut notamment les mots composés (\emph{allume-cigare} sera donc décomposé en
\emph{allume} et \emph{cigare}). J'ai également choisi de ne pas respecter la casse et de considérer que deux mots
capitalisés différemment sont identiques.

\item []

\end{description}

\end{document}

